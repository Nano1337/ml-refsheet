\subsection*{Ch8: Power Series}

\textbf{Definition}: Taylor series for $f(x)$ centered at $x = x_0$ is $\sum_{n=0}^{\infty} \frac{f^{(n)}(x_0)}{n!}(x-x_0)^n$.

\textbf{Steps to finding Taylor Polynomial}: Find each $f^{(n)}(x_0)$, then plug into Taylor series formula. Note: be careful of implicit 
differentiation $d/dx(y') = y''$ but $d/dx(siny) = y'cosy$ by chain rule. 

\textbf{Find Convergence Set}: 1. Ratio test $lim_{n \rightarrow \infty} |\frac{a_{n}}{a_{n+1}}| = \rho$. 2. $(x_0-\rho, x_0 + \rho)$.  

\textbf{Manipulating Power Series}: $f(x) + g(x) = \sum_{n=0}^\infty (a_n + b_n)(x-x_0)^n, \rho = min(\rho_f, \rho_g)$. 
Cauchy-Product: $f(x)g(x) = \sum_{n=0}^\infty (\sum_{k=0}^n a_kb_{n-k})(x-x_0)^n$. Derivative: $f'(x) = \sum_{n=1}^\infty na_n(x-x_0)^{n-1}$. 
Integral: $\int_0^x f(t)dt = \sum_{n=0}^\infty \frac{a_n}{n+1}(x-x_0)^{n+1}$.

\textbf{Alternate form of Ratio test}: $lim_{k \rightarrow \infty} |\frac{a_{2k}}{a_{2k+2}} = L$ then $\sum_k=0^\infty a_{2k}x^{2k} has \rho = \sqrt{L}$.
$lim_{k \rightarrow \infty} |\frac{a_{2k+1}}{a_{2k+3}} = L$ then $\sum_k=0^\infty a_{2k+1}x^{2k+1} has \rho = \sqrt{L}$.
\textbf{Analytic Fxns}: infinitely diff at point $x_0$ and neighborhood can be expressed by convergent power series $f(x) = \sum_{n=0}^\infty a_n(x-x_0)^n$

\textbf{Find sum of sums=0}: 1. var transform to same $x^n$ 2. Sum = 0 then $a_n = 0$. 

\textbf{Familiar Power Series}: $e^x = \sum_{n=0}^\infty \frac{x^n}{n!}, sinx = \sum_{n=0}^\infty (-1)^n\frac{x^{2n+1}}{(2n+1)!}, cosx = \sum_{n=0}^\infty (-1)^n\frac{x^{2n}}{(2n)!}$
$lnx = \sum_{n=1}^\infty (-1)^{n-1}\frac{(x-1)^n}{n}, 1/(1-x) = \sum_{n=0}^\infty x^n$

\textbf{Finding Power Series Soln}: 1. Check P, Q are analytic at $x_0$ (not singular). 2. $y(x) = \sum_{n=0}^\infty a_n(x-x_0)^n$ and diff to get $y', y''$ to plug into ODE. 
3. Var transform to get same $x^n$ 4. Set sum start to same point 5. Add summation together 6. Solve recurrence by writing out first few terms and sub to reduce to 
sum. 7. Find $\rho$ with ratio test. Note: even and odd can be lienarly independent. 

\textbf{Find $\rho$ lower bound}: 1. Find if P, Q have singular points 2. $\rho \geq$ distance to nearest singular point. For complex numbers, 
dist(a+bi, c+di) = $\sqrt{(a-c)^2 + (b-d)^2}$. If P, Q aren't singular then $\rho = \infty$.

\textbf{Power Series Expansion with Variable Coeffs}: 
1. Write out first few terms of variable coeff power series. 2. Expand up to $x^n$ where n is number of desired terms. 3. Collate each $x^n$ coeff sum = 0. 
4. Use IVP to progressively solve for each coeff. 5. Answer $y(x) = \sum_{n=0}^\infty a_n(x-x_0)^n$