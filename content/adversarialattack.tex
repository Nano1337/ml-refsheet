\subsection*{Adversarial Attacks}
White-Box: architecture and weights are known
Black-Box: only access to input and output (e.g. API)
Gray-Box: architecture known but not weights

We can attain an adversarial generally by adding some perturbation that maximizes the loss: 
$max_{\delta} \text{ } loss(\theta, x + \delta, y) $
We want $\delta$ that's small w.r.t.
$l_p$ norm,
Rotation/Translation,
VGG feature perturbation or any other perturbation

\textbf{White Box Attack Methods}

\textbf{Fast Gradient Sign Method (FGSM) Attack}: 
Use pretrained classifier like ResNet50: $\tilde y = f(\theta, x)$
Find adversarial example that maximizes the loss: $\mathcal{L} (x', y) = \mathcal{L}(f(\theta, x'), y)$ 
Bounded perturbation s.t.: $||x'-x||_{\infty}\leq \epsilon$ , where $\epsilon$ is attack strength
Optimal adversarial image: $x' = x + \epsilon sign(\nabla_{x}\mathcal{L}(x, y))$ 

\textbf{Iterative FGSM Attack}: 
Let m be number of iterations 
$x^{(m)} = x^{(m-1)} + \epsilon sign(\nabla_{x}\mathcal{L}(x^{(m-1)}, y))$ 
Both (I)FGSM are fix-perturbation attacks

\textbf{(Iterative) Least Likely Attack}: 
Similar to FSGM but $y_{LL}$ is the least likely (LL) class predicted by the network on clean image x
$x' = x - \epsilon sign(\nabla_{x}\mathcal{L}(x, y_{LL}))$
Strong attack as it emphasizes the least likely class

\textbf{Projected Gradient Descent Attack}: 
We can take a gradient step and then project it back to the feasible set $\Delta$ since the perturbed input may not lie on the data manifold (similar to denoising AutoEncoder logic)
$\delta := \mathcal{P}_{\Delta}[\delta + \nabla_{\delta}Loss(x + \delta), y; \theta]$ 
For example, the projected gradient descent applied to $l_\infty$ ball, repeat: 
$\delta := Clip_{\epsilon}[\delta + \alpha \nabla_{\delta}J(\delta)]$ 
Slower than FGSM but typically able to find better optima 

\textbf{Carlini and Wagner (CW) L2 Attack}: 
zero-confidence attack 
for all $t \neq y$ , find the adversarial image that will be classified by $t$ as solving the problem: $min_{\delta}||\delta||_{2}^{2}$ subject to $f(x + \delta) = y, x + \delta \in [0, 1]^{n}$
Finding the exact solution is difficult so we use relaxed version
$min_{\delta}||\delta||_{2}^{2} + c \cdot g(x + \delta)$  subject to $x + \delta \in [0, 1]^{n}, c \geq 0$ 
Let $Z(x)$ be the NN activations before the logit output layer, also called the embeddings
$g(x) = max\left(max_{i\neq t} (Z(x)_{i} - Z(x)_{t}), 0\right)$
Let $\delta = \frac{1}{2}(tanh(w) + 1) - x$
With the following constrained optimization problem: 
$
min_w||\frac{1}{2}(tanh(w)+1)-x||_{2}^{2} + c,
ReLU\{max_{i\neq t}Z\left(\left(\frac{1}{2}tanh(w) + 1\right)_{i}\right)- Z\left(\frac{1}{2}\left(tanh(w) + 1\right)\right)_{t}\}
$
Powerful attack method that resists many defenses

\textbf{Universal Adversarial Perturbation Attacks}: 
A single perturbation on any image with high probability (e.g. 0.8+)
Generalize well across different models

\textbf{Single Pixel Attack}:
self-explanatory. Only modify one pixel

\textbf{Poisoning Attacks}:
manipulating the training data itself rather than during inference
Maintain accuracy but hamper generalization due to outliers through poisoning

\subsection*{Adversarial Training}
Do training with adversarial samples
\textbf{Outer Minimization}:
$min_{\theta}\sum\limits_{x, y \in S} max_{\delta \in \Delta} Loss(x + \delta, y; \theta)$
We can do gradient descent, but how do we find the gradient of the inner maximization term? 

\textbf{Danskin's Theorem}: 
A fundamental result from optimization is: 
$\nabla_{\theta}max_{\delta \in \Delta} Loss(x + \delta, y; \theta) = \nabla_{\theta}Loss(x = \delta^{*}, y; \theta)$
where
$\delta^{*}=max_{\delta \in \Delta} Loss(x + \delta, y; \theta)$
which means we can optimize through the max by finding its maximum value, but this only applies when max is found exactly. 

\textbf{Steps}: 
1. Select mini-batch B
2. For each sample, calculate the corresponding adversarial sample 
3. Update parameters where learning rate is divided by batch size
It's also common to mix adversarial samples and normal samples for stability reasons

